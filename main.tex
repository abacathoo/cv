%%%%%%%%%%%%%%%%%%%%%%%%%%%%%%%%%%%%%%%%%%%%%%%%%%%%%%%%%%%%%%%%%%%%%%%%
%%%%%%%%%%%%%%%%%%%%%% Simple LaTeX CV Template %%%%%%%%%%%%%%%%%%%%%%%%
%%%%%%%%%%%%%%%%%%%%%%%%%%%%%%%%%%%%%%%%%%%%%%%%%%%%%%%%%%%%%%%%%%%%%%%%

%%%%%%%%%%%%%%%%%%%%%%%%%%%%%%%%%%%%%%%%%%%%%%%%%%%%%%%%%%%%%%%%%%%%%%%%
%% NOTE: If you find that it says                                     %%
%%                                                                    %%
%%                           1 of ??                                  %%
%%                                                                    %%
%% at the bottom of your first page, this means that the AUX file     %%
%% was not available when you ran LaTeX on this source. Simply RERUN  %%
%% LaTeX to get the ``??'' replaced with the number of the last page  %%
%% of the document. The AUX file will be generated on the first run   %%
%% of LaTeX and used on the second run to fill in all of the          %%
%% references.                                                        %%
%%%%%%%%%%%%%%%%%%%%%%%%%%%%%%%%%%%%%%%%%%%%%%%%%%%%%%%%%%%%%%%%%%%%%%%%

%%%%%%%%%%%%%%%%%%%%%%%%%%%% Document Setup %%%%%%%%%%%%%%%%%%%%%%%%%%%%

% Don't like 10pt? Try 11pt or 12pt
\documentclass[10pt]{article}

% This is a helpful package that puts math inside length specifications
\usepackage{calc}


% Simpler bibsection for CV sections
% (thanks to natbib for inspiration)
\makeatletter
\newlength{\bibhang}
\setlength{\bibhang}{1em}
\newlength{\bibsep}
 {\@listi \global\bibsep\itemsep \global\advance\bibsep by\parsep}
\newenvironment{bibsection}%
        {\vspace{-\baselineskip}\begin{list}{}{%
       \setlength{\leftmargin}{\bibhang}%
       \setlength{\itemindent}{-\leftmargin}%
       \setlength{\itemsep}{\bibsep}%
       \setlength{\parsep}{\z@}%
        \setlength{\partopsep}{0pt}%
        \setlength{\topsep}{0pt}}}
        {\end{list}\vspace{-.6\baselineskip}}
\makeatother

% Layout: Puts the section titles on left side of page
\reversemarginpar

%
%         PAPER SIZE, PAGE NUMBER, AND DOCUMENT LAYOUT NOTES:
%
% The next \usepackage line changes the layout for CV style section
% headings as marginal notes. It also sets up the paper size as either
% letter or A4. By default, letter was used. If A4 paper is desired,
% comment out the letterpaper lines and uncomment the a4paper lines.
%
% As you can see, the margin widths and section title widths can be
% easily adjusted.
%
% ALSO: Notice that the includefoot option can be commented OUT in order
% to put the PAGE NUMBER *IN* the bottom margin. This will make the
% effective text area larger.
%
% IF YOU WISH TO REMOVE THE ``of LASTPAGE'' next to each page number,
% see the note about the +LP and -LP lines below. Comment out the +LP
% and uncomment the -LP.
%
% IF YOU WISH TO REMOVE PAGE NUMBERS, be sure that the includefoot line
% is uncommented and ALSO uncomment the \pagestyle{empty} a few lines
% below.
%

%% Use these lines for letter-sized paper
%%\usepackage[paper=letterpaper,
%%            %includefoot, % Uncomment to put page number above margin
 %%           marginparwidth=1.2in,     % Length of section titles
%%            marginparsep=.05in,       % Space between titles and text
%%            margin=0.5in,               % 1 inch margins
%%            includemp]{geometry}

%% Use these lines for A4-sized paper
\usepackage[paper=a4paper,
            %includefoot, % Uncomment to put page number above margin
            marginparwidth=30.5mm,    % Length of section titles
           marginparsep=1.5mm,       % Space between titles and text
            margin=10mm,              % 25mm margins
            includemp]{geometry}

%% More layout: Get rid of indenting throughout entire document
\setlength{\parindent}{0in}

%% This gives us fun enumeration environments. compactitem will be nice.
\usepackage{paralist}

%% Reference the last page in the page number
%
% NOTE: comment the +LP line and uncomment the -LP line to have page
%       numbers without the ``of ##'' last page reference)
%
% NOTE: uncomment the \pagestyle{empty} line to get rid of all page
%       numbers (make sure includefoot is commented out above)
%
\usepackage{fancyhdr,lastpage}
\pagestyle{fancy}
\pagestyle{empty}      % Uncomment this to get rid of page numbers
\fancyhf{}\renewcommand{\headrulewidth}{0pt}
\fancyfootoffset{\marginparsep+\marginparwidth}
\newlength{\footpageshift}
\setlength{\footpageshift}
          {0.5\textwidth+0.5\marginparsep+0.5\marginparwidth-2in}
\lfoot{\hspace{\footpageshift}%
       \parbox{4in}{\, \hfill %
                    \arabic{page} of \protect\pageref*{LastPage} % +LP
%                    \arabic{page}                               % -LP
                    \hfill \,}}

% Finally, give us PDF bookmarks
\usepackage{color,hyperref}
\definecolor{darkblue}{rgb}{0.0,0.0,0.4}
\hypersetup{colorlinks,breaklinks,
            linkcolor=darkblue,urlcolor=darkblue,
            anchorcolor=darkblue,citecolor=darkblue}

%%%%%%%%%%%%%%%%%%%%%%%% End Document Setup %%%%%%%%%%%%%%%%%%%%%%%%%%%%


%%%%%%%%%%%%%%%%%%%%%%%%%%% Helper Commands %%%%%%%%%%%%%%%%%%%%%%%%%%%%

% The title (name) with a horizontal rule under it
%
% Usage: \makeheading{name}
%
% Place at top of document. It should be the first thing.
\newcommand{\makeheading}[1]%
        {\hspace*{-\marginparsep minus \marginparwidth}%
         \begin{minipage}[t]{\textwidth+\marginparwidth+\marginparsep}%
                {\large \bfseries #1}\\[-0.15\baselineskip]%
                 \rule{\columnwidth}{1pt}%
         \end{minipage}}

% The section headings
%
% Usage: \section{section name}
%
% Follow this section IMMEDIATELY with the first line of the section
% text. Do not put whitespace in between. That is, do this:
%
%       \section{My Information}
%       Here is my information.
%
% and NOT this:
%
%       \section{My Information}
%
%       Here is my information.
%
% Otherwise the top of the section header will not line up with the top
% of the section. Of course, using a single comment character (%) on
% empty lines allows for the function of the first example with the
% readability of the second example.
\renewcommand{\section}[2]%
        {\pagebreak[2]\vspace{1.3\baselineskip}%
         \phantomsection\addcontentsline{toc}{section}{#1}%
         \hspace{0in}%
         \marginpar{
         \raggedright \scshape #1}#2}

\renewcommand{\subsection}[2]%
        {\pagebreak[2]\vspace{1.3\baselineskip}%
         \phantomsection\addcontentsline{toc}{section}{#1}%
         \hspace{0in}%
         \marginpar{
         \raggedright \scshape \footnotesize #1}#2}

% An itemize-style list with lots of space between items
\newenvironment{outerlist}[1][\enskip\textbullet]%
        {\begin{itemize}[#1]}{\end{itemize}%
         \vspace{-.6\baselineskip}}

% An environment IDENTICAL to outerlist that has better pre-list spacing
% when used as the first thing in a \section
\newenvironment{lonelist}[1][\enskip\textbullet]%
        {\vspace{-\baselineskip}\begin{list}{#1}{%
        \hspace{1cm}\setlength{\partopsep}{0pt}%
        \setlength{\topsep}{0pt}}}
        {\end{list}\vspace{-.6\baselineskip}}

% An itemize-style list with little space between items
\newenvironment{innerlist}[1][\enskip\textbullet]%
        {\begin{compactitem}[#1]}{\end{compactitem}}

% An environment IDENTICAL to innerlist that has better pre-list spacing
% when used as the first thing in a \section
\newenvironment{loneinnerlist}[1][\enskip\textbullet]%
        {\vspace{-\baselineskip}\begin{compactitem}[#1]}
        {\end{compactitem}\vspace{-.6\baselineskip}}

% To add some paragraph space between lines.
% This also tells LaTeX to preferably break a page on one of these gaps
% if there is a needed pagebreak nearby.
\newcommand{\blankline}{\quad\pagebreak[2]}

% Uses hyperref to link DOI
\newcommand\doilink[1]{\href{http://dx.doi.org/#1}{#1}}
\newcommand\doi[1]{doi:\doilink{#1}}


%%%%%%%%%%%%%%%%%%%%%%%% End Helper Commands %%%%%%%%%%%%%%%%%%%%%%%%%%%

%%%%%%%%%%%%%%%%%%%%%%%%% Begin CV Document %%%%%%%%%%%%%%%%%%%%%%%%%%%%

\begin{document}
\makeheading{William Kenyon (British Citizen)}

\section{Contact Information}
%
% NOTE: Mind where the & separators and \\ breaks are in the following
%       table.
%
% ALSO: \rcollength is the width of the right column of the table
%       (adjust it to your liking; default is 1.85in).
%
\newlength{\rcollength}\setlength{\rcollength}{1.85in}%
%
\begin{tabular}[t]{@{}p{\textwidth-\rcollength}p{\rcollength}}
\href{http://www.emma.cam.ac.uk}{Emmanuel College} & \\
\href{http://www.cam.ac.uk}{Cambridge} & \textit{Mobile:} +447907808969 \\
CB2 3AP                                & \textit{E-mail:} \href{mailto:wk249@cam.ac.uk}{wk249@cam.ac.uk}\\
\end{tabular}



\section{Higher Education}
\href{http://www.cam.ac.uk/}{\textbf{Cambridge University}}, Cambridgeshire, UK\hfill\textbf{2009--Present}

\subsection{M.Eng. \newline Computer Science}
I have been accepted to start this course in October. There are strict conditions for entry, and I was accepted as my B.A. mark was 0.5\% below the cut-off for a $1^{st}$ class.

\subsection{B.A. \newline Computer Science \newline Class 2.1}\textbf{Taught Courses}
\begin{outerlist}
\item\textbf{Software Engineering}: C/C$++$, Java, ML, Prolog, System Verilog, The Dangers of Floating Point Arithmetic, Data Structures \& Algorithms, Development Methodology, Evolution of Programming Languages, Failures in Software Engineering, and Unix Tools.

\item\textbf{Systems}: Computer Architecture/Hardware/Networking, Concurrent/Distributed/Operating Systems, Digital Electronics and Digital Signal Processing.

\item\textbf{Theory}:  Complexity/Computation/Information Theory, Databases, Denotational/Operational Semantics, Regular Languages \& Finite Automauta, Specification \& Verification, and Types.

\item\textbf{Specialist}: Decompilation, Optimizing Compilers,  Computer Graphics, Natural Language Processing, Quantum Computing, and Security.

\item\textbf{Mathematics:} Calculus, Discrete Mathematics, Fourier Series/Transforms, Matricies, Probablility, Taylor Series, and Vectors.

\item\textbf{Physics:} Mechanics, Electromagnetism, Special Relativity, and Quantum Mechanics.

\item\textbf{Non-technical:} Buisness, E-commerce, Economics, Ethics, Law, and Philosophy.
\end{outerlist}
\vspace{0.5cm}
\textbf{Projects}

\begin{outerlist}
\item\textbf{Third Year Solo Project:} A state-of-the-art artificial intelligence technique was implemented for the first time in \href{http://www.haskell.org/haskellwiki/Introduction}{Haskell}. My dissertation, \href{https://github.com/abacathoo/dissertation/blob/master/latex/demodiss/diss.pdf?raw=true}{A Monte Carlo Tree Search Library in Haskell}, illustrated how use of \href{http://www.agiledata.org/images/tddSteps.jpg}{Test-Driven Development} led to a very stable product.

\item\textbf{Second Year Group Project:} A radio-DJ SMS hub was constructed for our client,  \href{http://frontlinesms.com}{FrontlineSMS}. I delivered a presentation \& demonstration of the project to a large audience of academics and local business people. The audience voted our project second out of fifteen groups.

%Worked in team of five, in which everyone wrote code managed under version control system. Project was for a real client and bi-weekly progress meetings were made. 

\item\textbf{Smaller Projects} 
\begin{innerlist}
\item Multi-player capture-the-flag agent implemented in \href{http://www.python.org/doc/essays/blurb.html}{Python}. 
\item \href{http://www.ponggame.org/}{Pong}, and \href{http://demos.sftrabbit.co.uk/game-of-life/}{Conway's Game of Life} were implemented on a \href{http://www.altera.com/products/fpga.html}{Field Programmable Gate Array} in \href{http://www.systemverilog.org/}{System Verilog} \& \href{http://logos.cs.uic.edu/366/notes/mips%20quick%20tutorial.htm}{MIPS Assembly Language}. 
\item Shortest-path-finder and \href{http://www.math.utah.edu/~pa/math/mandelbrot/large.gif}{Mandelbrot Set} renderer implemented in \href{http://www.smlnj.org/sml.html}{ML}. \item Internet chat server \& client implemented in \href{http://java.sun.com/docs/overviews/java/java-overview-1.html}{Java}. 
\item \href{http://www.siongboon.com/projects/2006-03-06_serial_communication/IP-Header-v4.png}{IP}/\href{http://www.siongboon.com/projects/2006-03-06_serial_communication/TCP-Header.png}{TCP} packet parser implemented in \href{http://en.wikipedia.org/wiki/C_(programming_language)}{C}. 
\item Various low-level electronics projects were implemented at the transistor and logic gate level.
\end{innerlist}
\end{outerlist}
\section{Secondary Education}\textbf{\href{http://www.queenelizabeth.cumbria.sch.uk/}{Queen Elizabeth School}}, Kirkby Lonsdale, Cumbria, UK \hfill \textbf{2002--2009}
\begin{innerlist}
%\item[] \textbf{GCE} Computing: \emph{A (598/600)}, Physics: \emph{A}, Maths: \emph{A}, Further Maths: \emph{A}. \hfill \textbf{2009}
%\item[] \textbf{GCE} Electronics: \emph{A (full marks in 5/6 modules)}\hfill \textbf{2008}
\item[] \textbf{GCE} Grade \emph{A} in Computing, Electronics, Physics, Maths, and Further Maths. \hfill \textbf{2009}
\item[] \textbf{GCSE} $9$\emph{A*}s and $4$\emph{A}s.\hfill \textbf{2007}

\end{innerlist}


\section{Related Employment}
%
\href{http://www.realvnc.com/}{\textbf{RealVNC Ltd.}},
Cambridgeshire, UK \hfill \textbf{Summer 2011}
\begin{outerlist}

\item[] \textit{Ported \href{http://www.realvnc.com/products/index.html}{VNCViewer} to the \href{http://wiki.tcl.tk/_natcl/balls.html}{Google Chrome Native Client (NaCl)} platform.}%
       
\begin{innerlist}
\item Rewrote large chunks of \href{http://www.unix.org/what_is_unix.html}{UNIX} specific C/C$++$ code to run in the NaCl sand-box.
\item Replaced network sockets with a \href{https://developer.mozilla.org/en/JavaScript/Guide/JavaScript_Overview}{Javascript} + \href{http://www.websocket.org/}{Websockets} tunnel.
\item Rewrote concurrency code to use \href{http://www.kernel.org/doc/man-pages/online/pages/man7/pthreads.7.html}{pthreads} rather than the UNIX \href{http://unixhelp.ed.ac.uk/CGI/man-cgi?select+2}{select} function.
\item Built an in-browser front-end for viewing the remote screen and emulating remote input devices.
%\item Project was working, with code thoroughly documented and commented by the time I left.
\item Full-time employee was brought on after I left to turn my project into a releasable product.

\end{innerlist}
\end{outerlist}

\blankline

\section{Related Employment}
%
\href{http://www.lancs.ac.uk/}{\textbf{Lancaster University}},
Lancashire, UK        \hfill \textbf{Summer 2009 \& 2010}
\begin{outerlist}


\item[] \textit{Worked with a research group on \href{http://www.p2p-next.org/}{p2pnext}, a peer-to-peer live telivision streaming project.}%
\begin{innerlist}
\item Modified the p2pnext streaming-bit-torrent client to favour geographically close peers.
\item Built a user support portal and request system for rolling out streaming set top boxes.
\end{innerlist}
\item[] \textit{Worked with a research group on a community \href{http://www.comp.lancs.ac.uk/resilience/project/wray-community-mesh-network}{Wireless Mesh Network} based on \href{http://wiki.openwrt.org/about/start}{OpenWRT}.}%
\begin{innerlist}
\item Built an access-control portal -- based on \href{http://dev.wifidog.org/wiki/FAQ}{Wifidog} -- to limit the people with access to the network.
\item Modfied source code of OpenWRT \& Wifidog  within my own development-build.
%\item Continued projects I had started in the previous summer using the documentation I had written.
%\item Installed the mesh boxes we had been working on all summer in a village and trained community volunteers to use the system and help others. Monitored usage using my monitoring software.

%\item Gained experiecnce of working in a team, working to deadlines, attending project meetings.
%\item Experience of documenting thoroughly and producing code consistent with rest of team.
\item Installed the physical mesh-boxes on roof-tops in the local community.
%\item Developed a log system \& web-interface to inspect the status of individual mesh-boxes.
\item Developed a system based on the \href{http://en.wikipedia.org/wiki/Preboot_Execution_Environment}{Pre-boot Execution Environment} to re-flash `bricked' mesh-boxes. This eliminated the need to climb onto roof-tops and replace the firmware memory.
\end{innerlist}
\end{outerlist}



%\textbf{\href{http://www.edfenergy.com}{EDF Energy}, Heysham II Nuclear Power Station}, Lancashire, UK\hfill \textbf{Summer 2008}
%\begin{outerlist}

%\item[] \textit{Work Experience}% **??? you arent saying anything. what was your title. this is already headed uder related employment. 
        
%\begin{innerlist}
%\item Gained understanding of how computer systems are used to monitor and control a nuclear reactor.
%\item Wrote code on development system (carbon copy of the computer system controlling the reactor). **this does not say anything to me. so what if it is a carbon copy. what good did you do? 

%\end{innerlist}

%\end{outerlist}
\section{Other Employment}
%
\textbf{\href{http://www.queenelizabeth.cumbria.sch.uk/}{Queen Elizabeth School}}, Kirkby Lonsdale, Cumbria, UK \hfill \textbf{2007--2009}

\par \textbf{\href{http://www.bridgehousefarm.co.uk/}{Bridge House Farm Tearooms}}, Wray, Lancaster, Lancashire, UK \hfill \textbf{2005--2008}
\begin{outerlist}

\item[] \textit{Part time cleaning and `waiting-on' jobs.} 
\begin{innerlist}
\item Due to good work ethic and good table-side manners, I was given jobs of greater responsibility.
\end{innerlist}

\end{outerlist}

\subsection{\normalsize Skills \footnotesize Programming Languages}
\textbf{General}: \href{http://en.wikipedia.org/wiki/C_(programming_language)}{C}/\href{http://www.cplusplus.com/info/description/}{C$++$}, \href{http://www.haskell.org/haskellwiki/Introduction}{Haskell}, \href{http://java.sun.com/docs/overviews/java/java-overview-1.html}{Java}, \href{http://www.smlnj.org/sml.html}{ML}, \href{http://www.cse.unsw.edu.au/~billw/cs9414/notes/prolog/intro.html}{Prolog}, and \href{http://en.wikipedia.org/wiki/Visual_Basic}{Visual Basic} .

\textbf{Hardware Description}: \href{http://www.systemverilog.org/}{System Verilog}.

\textbf{UNIX Tools}: \href{http://www.gnu.org/software/bash/manual/html_node/What-is-Bash_003f.html}{Bash}, \href{http://cimss.ssec.wisc.edu/wxwise/class/aos340/spr00/whatismatlab.htm}{Matlab}, \href{http://perldoc.perl.org/perlfaq1.html}{Perl}.

\textbf{Web:} \href{https://developer.mozilla.org/en/JavaScript/Guide/JavaScript_Overview}{Javascript}, \href{http://en.wikipedia.org/wiki/HTML}{HTML}, \href{http://en.wikipedia.org/wiki/PHP}{PHP}, and \href{http://en.wikipedia.org/wiki/SQL}{SQL}.

\textbf{Historical} (basic knowledge only): \href{http://en.wikipedia.org/wiki/ALGOL}{Algol}, \href{http://en.wikipedia.org/wiki/Fortran}{Fortran},  and \href{http://en.wikipedia.org/wiki/Smalltalk}{Smalltalk}. 

I studied programming languages formally in \href{http://www.cl.cam.ac.uk/teaching/1112/Types/}{Type Theory} and \href{http://www.cl.cam.ac.uk/teaching/1112/Semantics/}{Operational}/\href{http://www.cl.cam.ac.uk/teaching/1112/DenotSem/}{Denotational} Semantics. 

%\subsection{Code Style \& Documentation}
%I am versatile, and have can quickly adopt the coding/documentation style used by a particular company. I document code as I write it, and am used to using documentation generation tools such as \href{junit.com}{JUnit}.

\subsection{Libraries \& Tools}
\textbf{Concurrency:} Built-in Java concurrency, and \href{http://www.kernel.org/doc/man-pages/online/pages/man7/pthreads.7.html}{pthreads} in C/C$++$.

\textbf{Networking:} Built-in Java sockets, and \href{http://unixhelp.ed.ac.uk/CGI/man-cgi?select+2}{sockets} in C/C$++$.

\textbf{General:} Standard C/C$++$ library, \href{http://www.sgi.com/tech/stl/stl_introduction.html}{Standard Template Library}.

\textbf{Version Control:} \href{http://eagain.net/articles/git-for-computer-scientists/}{Git}, and \href{http://svnbook.red-bean.com/en/1.6/svn.intro.whatis.html}{Subversion}.

\textbf{Automated Documentation Generation:} \href{http://www.haskell.org/haskellwiki/Haddock}{Haddock}, and \href{http://www.devx.com/tips/Tip/13579}{JavaDoc}.

\textbf{Automated Unit Testing:} \href{http://junit.sourceforge.net/doc/faq/faq.htm#overview_1}{JUnit}, and \href{http://www.cse.chalmers.se/~rjmh/QuickCheck/manual_body.html}{QuickCheck}.



\textbf{Web:} \href{http://en.wikipedia.org/wiki/Ajax_(programming)}{Ajax}, \href{http://drupal.org/about/}{Drupal}, \href{http://nodejs.org/}{node.js}, and \href{http://www.websocket.org/}{Websocket}. 

\textbf{Operating Systems:} \href{http://en.wikipedia.org/wiki/Linux}{Linux}, \href{http://en.wikipedia.org/wiki/Macintosh}{Macintosh}, and \href{http://en.wikipedia.org/wiki/Windows}{Windows}. 

\subsection{Data Structures \& Algorithms}
\href{http://www-users.cs.umn.edu/~janardan/FHEAP/}{Fibonacci Heaps}, \href{http://en.wikipedia.org/wiki/Disjoint-set_data_structure}{Disjoint Sets}, \href{http://www.bluerwhite.org/btree/}{B Trees}, \href{http://en.wikipedia.org/wiki/Red–black_tree}{Red Black Trees}, \href{http://en.wikipedia.org/wiki/External_sorting}{External $n$-way Merge Sort}, together with basic data structures \& algorithms, were studied during my undergraduate degree. I have implemented many of the above in C$++$ or Java. I can find asymptotic complexity bounds (expressed in big \emph{O}, $\Omega$ and $\Theta$ notation) for algorithms using standard and amortized analysis.

\subsection{Logic \& Discrete Mathematics}
I can prove statements, formally/informally in first/second/higher order logics, temporal logics, hoare/sep\-aration logic, polymorphic lambda calculi and sequent calculi. This formal background gives me the mind-set required to write good code.

%\subsection{Specification \& Verification}
%I can write specifications for program code, and prove** the code meets the specifications. I can model small software/hardware systems, and proove that the model meets a specification written in Computation Tree Logic, Linear Time Logic or PSL.

\section{\footnotesize Teamwork}
Teamworking skills were essential at my internships, and during my undergraduate group project. I had client meetings, deadlines to meet, coding/documentation styles to adhere to, and presentations to give. My ability to communicate effectively is one of my greatest assets. 

%\subsection{Law}
%I have good knowledge of the laws and regulations relevant to the IT industry***. This was obtained during my undergraduate degree and internships***.

\section{Intrests}
%
I am a senior member of \href{http://www.culrc.com}{Cambridge University Lightweight Rowing Club}. The training regime is intense, and currently consists of 2 or 3 sessions per day. In 2012/2013, I shall be trialling for the Great Britain under-23 team.
I have also participated fully in society life, having skied (competitively), sailed (competitively), and sung my way through university.

%Ski at a local dry ski slope and participate in \href{www.ersa.com}{Eastern Region} races.**

%Sail a \emph{Laser I} dinghy competitively in weekly club races at a local sailing club.



%Have sung in \href{http://www.emmamusic.co.uk/index.php?page=chorus}{Emmanuel College Chorus}, performing in frequent concerts  and charity carols on street.

%Performed in \href{http://cambridge.gands.org.uk/}{Cambridge University Gilbert and Sullivan Society} operettas.**

%Danced with the \href{http://www.cambridgedancers.org/}{Cambridge Dancers Club}.**


\section{Referees}
%
\href{http://realvnc.com}{Hannah Clear} (\href{mailto:hannah.clear@realvnc.com}{hannah.clear@realvnc.com})

\href{http://www.cl.cam.ac.uk/~nad10}{Neil Dodgson} (\href{mailto:nad10@cam.ac.uk}{nad10@cam.ac.uk})

\href{http://www.comp.lancs.ac.uk/department/staff.php?name=race}{Nicholas Race} (\href{mailto:n.race@lancaster.ac.uk}{n.race@lancaster.ac.uk})

\href{}{Neil Read} (\href{mailto:neal.read@edf-energy.com}{neal.read@edf-energy.com})

\end{document}

%%%%%%%%%%%%%%%%%%%%%%%%%% End CV Document %%%%%%%%%%%%%%%%%%%%%%%%%%%%%
