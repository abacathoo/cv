%%%%%%%%%%%%%%%%%%%%%%%%%%%%%%%%%%%%%%%%%%%%%%%%%%%%%%%%%%%%%%%%%%%%%%%%
%%%%%%%%%%%%%%%%%%%%%% Simple LaTeX CV Template %%%%%%%%%%%%%%%%%%%%%%%%
%%%%%%%%%%%%%%%%%%%%%%%%%%%%%%%%%%%%%%%%%%%%%%%%%%%%%%%%%%%%%%%%%%%%%%%%

%%%%%%%%%%%%%%%%%%%%%%%%%%%%%%%%%%%%%%%%%%%%%%%%%%%%%%%%%%%%%%%%%%%%%%%%
%% NOTE: If you find that it says                                     %%
%%                                                                    %%
%%                           1 of ??                                  %%
%%                                                                    %%
%% at the bottom of your first page, this means that the AUX file     %%
%% was not available when you ran LaTeX on this source. Simply RERUN  %%
%% LaTeX to get the ``??'' replaced with the number of the last page  %%
%% of the document. The AUX file will be generated on the first run   %%
%% of LaTeX and used on the second run to fill in all of the          %%
%% references.                                                        %%
%%%%%%%%%%%%%%%%%%%%%%%%%%%%%%%%%%%%%%%%%%%%%%%%%%%%%%%%%%%%%%%%%%%%%%%%

%%%%%%%%%%%%%%%%%%%%%%%%%%%% Document Setup %%%%%%%%%%%%%%%%%%%%%%%%%%%%

% Don't like 10pt? Try 11pt or 12pt
\documentclass[10pt]{article}

% This is a helpful package that puts math inside length specifications
\usepackage{calc}


% Simpler bibsection for CV sections
% (thanks to natbib for inspiration)
\makeatletter
\newlength{\bibhang}
\setlength{\bibhang}{1em}
\newlength{\bibsep}
 {\@listi \global\bibsep\itemsep \global\advance\bibsep by\parsep}
\newenvironment{bibsection}%
        {\vspace{-\baselineskip}\begin{list}{}{%
       \setlength{\leftmargin}{\bibhang}%
       \setlength{\itemindent}{-\leftmargin}%
       \setlength{\itemsep}{\bibsep}%
       \setlength{\parsep}{\z@}%
        \setlength{\partopsep}{0pt}%
        \setlength{\topsep}{0pt}}}
        {\end{list}\vspace{-.6\baselineskip}}
\makeatother

% Layout: Puts the section titles on left side of page
\reversemarginpar

%
%         PAPER SIZE, PAGE NUMBER, AND DOCUMENT LAYOUT NOTES:
%
% The next \usepackage line changes the layout for CV style section
% headings as marginal notes. It also sets up the paper size as either
% letter or A4. By default, letter was used. If A4 paper is desired,
% comment out the letterpaper lines and uncomment the a4paper lines.
%
% As you can see, the margin widths and section title widths can be
% easily adjusted.
%
% ALSO: Notice that the includefoot option can be commented OUT in order
% to put the PAGE NUMBER *IN* the bottom margin. This will make the
% effective text area larger.
%
% IF YOU WISH TO REMOVE THE ``of LASTPAGE'' next to each page number,
% see the note about the +LP and -LP lines below. Comment out the +LP
% and uncomment the -LP.
%
% IF YOU WISH TO REMOVE PAGE NUMBERS, be sure that the includefoot line
% is uncommented and ALSO uncomment the \pagestyle{empty} a few lines
% below.
%

%% Use these lines for letter-sized paper
%%\usepackage[paper=letterpaper,
%%            %includefoot, % Uncomment to put page number above margin
 %%           marginparwidth=1.2in,     % Length of section titles
%%            marginparsep=.05in,       % Space between titles and text
%%            margin=0.5in,               % 1 inch margins
%%            includemp]{geometry}

%% Use these lines for A4-sized paper
\usepackage[paper=a4paper,
            %includefoot, % Uncomment to put page number above margin
            marginparwidth=30.5mm,    % Length of section titles
           marginparsep=1.5mm,       % Space between titles and text
            margin=10mm,              % 25mm margins
            includemp]{geometry}

%% More layout: Get rid of indenting throughout entire document
\setlength{\parindent}{0in}

%% This gives us fun enumeration environments. compactitem will be nice.
\usepackage{paralist}

%% Reference the last page in the page number
%
% NOTE: comment the +LP line and uncomment the -LP line to have page
%       numbers without the ``of ##'' last page reference)
%
% NOTE: uncomment the \pagestyle{empty} line to get rid of all page
%       numbers (make sure includefoot is commented out above)
%
\usepackage{fancyhdr,lastpage}
\pagestyle{fancy}
\pagestyle{empty}      % Uncomment this to get rid of page numbers
\fancyhf{}\renewcommand{\headrulewidth}{0pt}
\fancyfootoffset{\marginparsep+\marginparwidth}
\newlength{\footpageshift}
\setlength{\footpageshift}
          {0.5\textwidth+0.5\marginparsep+0.5\marginparwidth-2in}
\lfoot{\hspace{\footpageshift}%
       \parbox{4in}{\, \hfill %
                    \arabic{page} of \protect\pageref*{LastPage} % +LP
%                    \arabic{page}                               % -LP
                    \hfill \,}}

% Finally, give us PDF bookmarks
\usepackage{color,hyperref}
\definecolor{darkblue}{rgb}{0.0,0.0,0.3}
\hypersetup{colorlinks,breaklinks,
            linkcolor=darkblue,urlcolor=darkblue,
            anchorcolor=darkblue,citecolor=darkblue}

%%%%%%%%%%%%%%%%%%%%%%%% End Document Setup %%%%%%%%%%%%%%%%%%%%%%%%%%%%


%%%%%%%%%%%%%%%%%%%%%%%%%%% Helper Commands %%%%%%%%%%%%%%%%%%%%%%%%%%%%

% The title (name) with a horizontal rule under it
%
% Usage: \makeheading{name}
%
% Place at top of document. It should be the first thing.
\newcommand{\makeheading}[1]%
        {\hspace*{-\marginparsep minus \marginparwidth}%
         \begin{minipage}[t]{\textwidth+\marginparwidth+\marginparsep}%
                {\large \bfseries #1}\\[-0.15\baselineskip]%
                 \rule{\columnwidth}{1pt}%
         \end{minipage}}

% The section headings
%
% Usage: \section{section name}
%
% Follow this section IMMEDIATELY with the first line of the section
% text. Do not put whitespace in between. That is, do this:
%
%       \section{My Information}
%       Here is my information.
%
% and NOT this:
%
%       \section{My Information}
%
%       Here is my information.
%
% Otherwise the top of the section header will not line up with the top
% of the section. Of course, using a single comment character (%) on
% empty lines allows for the function of the first example with the
% readability of the second example.
\renewcommand{\section}[2]%
        {\pagebreak[2]\vspace{1.3\baselineskip}%
         \phantomsection\addcontentsline{toc}{section}{#1}%
         \hspace{0in}%
         \marginpar{
         \raggedright \scshape #1}#2}

% An itemize-style list with lots of space between items
\newenvironment{outerlist}[1][\enskip\textbullet]%
        {\begin{itemize}[#1]}{\end{itemize}%
         \vspace{-.6\baselineskip}}

% An environment IDENTICAL to outerlist that has better pre-list spacing
% when used as the first thing in a \section
\newenvironment{lonelist}[1][\enskip\textbullet]%
        {\vspace{-\baselineskip}\begin{list}{#1}{%
        \setlength{\partopsep}{0pt}%
        \setlength{\topsep}{0pt}}}
        {\end{list}\vspace{-.6\baselineskip}}

% An itemize-style list with little space between items
\newenvironment{innerlist}[1][\enskip\textbullet]%
        {\begin{compactitem}[#1]}{\end{compactitem}}

% An environment IDENTICAL to innerlist that has better pre-list spacing
% when used as the first thing in a \section
\newenvironment{loneinnerlist}[1][\enskip\textbullet]%
        {\vspace{-\baselineskip}\begin{compactitem}[#1]}
        {\end{compactitem}\vspace{-.6\baselineskip}}

% To add some paragraph space between lines.
% This also tells LaTeX to preferably break a page on one of these gaps
% if there is a needed pagebreak nearby.
\newcommand{\blankline}{\quad\pagebreak[2]}

% Uses hyperref to link DOI
\newcommand\doilink[1]{\href{http://dx.doi.org/#1}{#1}}
\newcommand\doi[1]{doi:\doilink{#1}}


%%%%%%%%%%%%%%%%%%%%%%%% End Helper Commands %%%%%%%%%%%%%%%%%%%%%%%%%%%

%%%%%%%%%%%%%%%%%%%%%%%%% Begin CV Document %%%%%%%%%%%%%%%%%%%%%%%%%%%%

\begin{document}
\makeheading{William Kenyon}

\section{Contact Information}
%
% NOTE: Mind where the & separators and \\ breaks are in the following
%       table.
%
% ALSO: \rcollength is the width of the right column of the table
%       (adjust it to your liking; default is 1.85in).
%
\newlength{\rcollength}\setlength{\rcollength}{1.85in}%
%
\begin{tabular}[t]{@{}p{\textwidth-\rcollength}p{\rcollength}}
\href{http://www.emma.cam.ac.uk}{Emmanuel College} & \\
\href{http://www.cam.ac.uk}{Cambridge} & \textit{Mobile:} +447907808969 \\
CB2 3AP                                & \textit{E-mail:} \href{mailto:wk249@cam.ac.uk}{wk249@cam.ac.uk}\\
\end{tabular}

\section{Education}
%
\href{http://www.cam.ac.uk/}{\textbf{Cambridge University}}, Cambs, UK
\begin{outerlist}
\item[] B.A., \href{http://www.cl.cam.ac.uk/}{Computer Science Tripos}, Part Ia \hfill \textbf{June 2010}
        \begin{innerlist}
        \item Computer Science Paper I (\emph{Class II, division 2}): Functional Programming, Algorithms, Logic.
        \item Computer Science Paper II (\emph{Class II, division 1}): Systems, Digital Electronics, Logic II.
        \item Mathematics (\emph{Class II, division 1}): Probability, Calculus, Fourier Series, Matricies.
	\item Physics (\emph{Class II, division 2}): Mechanics, Electromagnetism, Relativity, Quantum Mechanics.
        \end{innerlist}
\item[] Starting Ib Group Project on software for SMS messaging in third world countries.
\end{outerlist}

\blankline

\textbf{\href{http://www.queenelizabeth.cumbria.sch.uk/}{Queen Elizabeth School}}, Kirkby Lonsdale, Cumb, UK
\begin{innerlist}
\item[] \textbf{GCE} Computing: \emph{A (598/600)}; Physics: \emph{A}; Maths: \emph{A}; Further Maths: \emph{A}. \hfill \textbf{June 2009}
\item[] \textbf{GCE} Electronics: \emph{A (full marks in 5/6 modules, one of the AS modules not taken)}\hfill \textbf{June 2008}
\item[] \textbf{GCSE} \emph{9A*; 4A.}\hfill \textbf{June 2007}

\end{innerlist}


\section{Employment}
%
\href{http://www.lancs.ac.uk/}{\textbf{Lancaster University}},
Lancs, UK
\begin{outerlist}

\item[] \textit{Invited back to work with same research group as previous summer}%
        \hfill \textbf{June 2010 to October 2010}
\begin{innerlist}
\item Continued projects I had started in the previous summer using the documentation I had written.
\item Installed the mesh boxes we had been working on all summer in a village and trained community volunteers to use the system and help others. Monitored usage using my monitoring software.
\item Built user support portal and request system + release pattern for rolling out set top boxes.
\item Worked on live streaming technologies over BitTorrent. (\href{http://p2p-next.org}{p2pnext}).
\end{innerlist}

\item[] \textit{Summer work after secondary school with research group}%
        \hfill \textbf{June 2009 to October 2009}
\begin{innerlist}	                                      
\item Gained experiecnce of working in a team, working to deadlines, attending project meetings.
\item Experience of documenting thoroughly and producing code consistent with rest of team.
\item Built and maintained a test mesh network which was physically installed on coat stands in office.
\item Developed log system and web interface to inspect the status of the boxes on the mesh network.
\item Developed remote system using PXE to reflash bricked boxes.

\end{innerlist}


\end{outerlist}


\blankline

\textbf{\href{http://www.british-energy.com/}{British Energy} (now \href{http://www.edfenergy.com}{EDF Energy}), Heysham II Nuclear Power Station}, Lancs, UK
\begin{outerlist}

\item[] \textit{Work Experience}%
        \hfill \textbf{June 2008}
\begin{innerlist}
\item Gained understanding of how computer systems are used to monitor and control a nuclear reactor.
\item Wrote some code on a test bed system (that uses identical hardware).

\end{innerlist}

\end{outerlist}

\section{Technical \linebreak Skills}
%
\textbf{Programming Languages}
\begin{innerlist}
\item Fluent: C, C$+$$+$, Java, JavaScript, ML, PHP, UNIX shell scripting, SQL
\item Good Knowledge: Haskell, Perl, Prolog, Python, SystemVerilog 
\end{innerlist}
\textbf{Tools}
\begin{innerlist}
\item Computer Applications: \TeX{}, vim, common productivity suites for linux/windows.
\item Operating Systems: Microsoft Windows, Linux.
\item Discrete Maths, Knowledge of low level hardware.
\end{innerlist}
\section{Extra Curricular}
%
\textbf{Sport}
\begin{innerlist}
 \item Rowing: I currently row in Emmanuel College senior 1st boat
 \item Cycling: I cycle between Cambridge and my home in the lake district at the start and end of each term. I carry my things I need to have at home in a backpack and four saddle bags.
 \item Skiing: I ski weekly at a local dry ski slope and participate in local races.
 \item Sailing: I own a `Laser I' dinghy and participate in weekly club races at a local sailing club.
\end{innerlist}

\textbf{Arts}
\begin{innerlist}
 \item Singing: I sing in the college chorus, performing in frequent concerts / charity carols on street.
 \item Musical Theatre: I am a member of the G\&S society and perform in operettas with them.
 \item Dancing: I take ballroom dancing lessons.
\end{innerlist}

\section{References}
\href{http://www.cl.cam.ac.uk/~nad10}{Neil Dodgson} (\href{mailto:nad10@cam.ac.uk}{nad10@cam.ac.uk});
\href{http://www.comp.lancs.ac.uk/department/staff.php?name=race}{Nicholas Race} (\href{mailto:n.race@lancaster.ac.uk}{n.race@lancaster.ac.uk}, +441524510123);
\href{}{Neil Read} (\href{mailto:neal.read@british-energy.com}{neal.read@british-energy.com}).

\end{document}

%%%%%%%%%%%%%%%%%%%%%%%%%% End CV Document %%%%%%%%%%%%%%%%%%%%%%%%%%%%%
