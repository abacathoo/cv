%%%%%%%%%%%%%%%%%%%%%%%%%%%%%%%%%%%%%%%%%%%%%%%%%%%%%%%%%%%%%%%%%%%%%%%%
%%%%%%%%%%%%%%%%%%%%%% Simple LaTeX CV Template %%%%%%%%%%%%%%%%%%%%%%%%
%%%%%%%%%%%%%%%%%%%%%%%%%%%%%%%%%%%%%%%%%%%%%%%%%%%%%%%%%%%%%%%%%%%%%%%%

%%%%%%%%%%%%%%%%%%%%%%%%%%%%%%%%%%%%%%%%%%%%%%%%%%%%%%%%%%%%%%%%%%%%%%%%
%% NOTE: If you find that it says                                     %%
%%                                                                    %%
%%                           1 of ??                                  %%
%%                                                                    %%
%% at the bottom of your first page, this means that the AUX file     %%
%% was not available when you ran LaTeX on this source. Simply RERUN  %%
%% LaTeX to get the ``??'' replaced with the number of the last page  %%
%% of the document. The AUX file will be generated on the first run   %%
%% of LaTeX and used on the second run to fill in all of the          %%
%% references.                                                        %%
%%%%%%%%%%%%%%%%%%%%%%%%%%%%%%%%%%%%%%%%%%%%%%%%%%%%%%%%%%%%%%%%%%%%%%%%

%%%%%%%%%%%%%%%%%%%%%%%%%%%% Document Setup %%%%%%%%%%%%%%%%%%%%%%%%%%%%

% Don't like 10pt? Try 11pt or 12pt
\documentclass[12pt]{article}

% This is a helpful package that puts math inside length specifications
\usepackage{calc}


% Simpler bibsection for CV sections
% (thanks to natbib for inspiration)
\makeatletter
\newlength{\bibhang}
\setlength{\bibhang}{1em}
\newlength{\bibsep}
 {\@listi \global\bibsep\itemsep \global\advance\bibsep by\parsep}
\newenvironment{bibsection}%
        {\vspace{-\baselineskip}\begin{list}{}{%
       \setlength{\leftmargin}{\bibhang}%
       \setlength{\itemindent}{-\leftmargin}%
       \setlength{\itemsep}{\bibsep}%
       \setlength{\parsep}{\z@}%
        \setlength{\partopsep}{0pt}%
        \setlength{\topsep}{0pt}}}
        {\end{list}\vspace{-.6\baselineskip}}
\makeatother

% Layout: Puts the section titles on left side of page
\reversemarginpar

%
%         PAPER SIZE, PAGE NUMBER, AND DOCUMENT LAYOUT NOTES:
%
% The next \usepackage line changes the layout for CV style section
% headings as marginal notes. It also sets up the paper size as either
% letter or A4. By default, letter was used. If A4 paper is desired,
% comment out the letterpaper lines and uncomment the a4paper lines.
%
% As you can see, the margin widths and section title widths can be
% easily adjusted.
%
% ALSO: Notice that the includefoot option can be commented OUT in order
% to put the PAGE NUMBER *IN* the bottom margin. This will make the
% effective text area larger.
%
% IF YOU WISH TO REMOVE THE ``of LASTPAGE'' next to each page number,
% see the note about the +LP and -LP lines below. Comment out the +LP
% and uncomment the -LP.
%
% IF YOU WISH TO REMOVE PAGE NUMBERS, be sure that the includefoot line
% is uncommented and ALSO uncomment the \pagestyle{empty} a few lines
% below.
%

%% Use these lines for letter-sized paper
%%\usepackage[paper=letterpaper,
%%            %includefoot, % Uncomment to put page number above margin
 %%           marginparwidth=1.2in,     % Length of section titles
%%            marginparsep=.05in,       % Space between titles and text
%%            margin=0.5in,               % 1 inch margins
%%            includemp]{geometry}

%% Use these lines for A4-sized paper
\usepackage[paper=a4paper,
            %includefoot, % Uncomment to put page number above margin
            marginparwidth=30.5mm,    % Length of section titles
           marginparsep=1.5mm,       % Space between titles and text
            margin=10mm,              % 25mm margins
            includemp]{geometry}

%% More layout: Get rid of indenting throughout entire document
\setlength{\parindent}{0in}

%% This gives us fun enumeration environments. compactitem will be nice.
\usepackage{paralist}

%% Reference the last page in the page number
%
% NOTE: comment the +LP line and uncomment the -LP line to have page
%       numbers without the ``of ##'' last page reference)
%
% NOTE: uncomment the \pagestyle{empty} line to get rid of all page
%       numbers (make sure includefoot is commented out above)
%
\usepackage{fancyhdr,lastpage}
\pagestyle{fancy}
\pagestyle{empty}      % Uncomment this to get rid of page numbers
\fancyhf{}\renewcommand{\headrulewidth}{0pt}
\fancyfootoffset{\marginparsep+\marginparwidth}
\newlength{\footpageshift}
\setlength{\footpageshift}
          {0.5\textwidth+0.5\marginparsep+0.5\marginparwidth-2in}
\lfoot{\hspace{\footpageshift}%
       \parbox{4in}{\, \hfill %
                    \arabic{page} of \protect\pageref*{LastPage} % +LP
%                    \arabic{page}                               % -LP
                    \hfill \,}}

% Finally, give us PDF bookmarks
\usepackage{color,hyperref}
\definecolor{darkblue}{rgb}{0.0,0.0,0.4}
\definecolor{gray}{rgb}{0.5,0.5,0.5}
\hypersetup{colorlinks,breaklinks,
            linkcolor=darkblue,urlcolor=darkblue,
            anchorcolor=darkblue,citecolor=darkblue}

%%%%%%%%%%%%%%%%%%%%%%%% End Document Setup %%%%%%%%%%%%%%%%%%%%%%%%%%%%


%%%%%%%%%%%%%%%%%%%%%%%%%%% Helper Commands %%%%%%%%%%%%%%%%%%%%%%%%%%%%

% The title (name) with a horizontal rule under it
%
% Usage: \makeheading{name}
%
% Place at top of document. It should be the first thing.
\newcommand{\makeheading}[1]%
        {\hspace*{-\marginparsep minus \marginparwidth}%
         \begin{minipage}[t]{\textwidth+\marginparwidth+\marginparsep}%
                {\large \bfseries #1}\\[-0.15\baselineskip]%
                 \rule{\columnwidth}{1pt}%
         \end{minipage}}

% The section headings
%
% Usage: \section{section name}
%
% Follow this section IMMEDIATELY with the first line of the section
% text. Do not put whitespace in between. That is, do this:
%
%       \section{My Information}
%       Here is my information.
%
% and NOT this:
%
%       \section{My Information}
%
%       Here is my information.
%
% Otherwise the top of the section header will not line up with the top
% of the section. Of course, using a single comment character (%) on
% empty lines allows for the function of the first example with the
% readability of the second example.
\renewcommand{\section}[2]%
        {\pagebreak[2]\vspace{1.3\baselineskip}%
         \phantomsection\addcontentsline{toc}{section}{#1}%
         \hspace{0in}%
         \marginpar{
         \raggedright \scshape #1}#2}

\renewcommand{\subsection}[2]%
        {\pagebreak[2]\vspace{1.3\baselineskip}%
         \phantomsection\addcontentsline{toc}{section}{#1}%
         \hspace{0in}%
         \marginpar{
         \raggedright \scshape \footnotesize #1}#2}

% An itemize-style list with lots of space between items
\newenvironment{outerlist}[1][\enskip\textbullet]%
        {\begin{itemize}[#1]}{\end{itemize}%
         \vspace{-.6\baselineskip}}

% An environment IDENTICAL to outerlist that has better pre-list spacing
% when used as the first thing in a \section
\newenvironment{lonelist}[1][\enskip\textbullet]%
        {\vspace{-\baselineskip}\begin{list}{#1}{%
        \hspace{1cm}\setlength{\partopsep}{0pt}%
        \setlength{\topsep}{0pt}}}
        {\end{list}\vspace{-.6\baselineskip}}

% An itemize-style list with little space between items
\newenvironment{innerlist}[1][\enskip\textbullet]%
        {\begin{compactitem}[#1]}{\end{compactitem}}

% An environment IDENTICAL to innerlist that has better pre-list spacing
% when used as the first thing in a \section
\newenvironment{loneinnerlist}[1][\enskip\textbullet]%
        {\vspace{-\baselineskip}\begin{compactitem}[#1]}
        {\end{compactitem}\vspace{-.6\baselineskip}}

% To add some paragraph space between lines.
% This also tells LaTeX to preferably break a page on one of these gaps
% if there is a needed pagebreak nearby.
\newcommand{\blankline}{\quad\pagebreak[2]}

% Uses hyperref to link DOI
\newcommand\doilink[1]{\href{http://dx.doi.org/#1}{#1}}
\newcommand\doi[1]{doi:\doilink{#1}}


%%%%%%%%%%%%%%%%%%%%%%%% End Helper Commands %%%%%%%%%%%%%%%%%%%%%%%%%%%

%%%%%%%%%%%%%%%%%%%%%%%%% Begin CV Document %%%%%%%%%%%%%%%%%%%%%%%%%%%%

\begin{document}
\makeheading{William Kenyon (M.Eng, B.A, British Citizen)}

\section{Contact Information}
%
% NOTE: Mind where the & separators and \\ breaks are in the following
%       table.
%
% ALSO: \rcollength is the width of the right column of the table
%       (adjust it to your liking; default is 1.85in).
%
\newlength{\rcollength}\setlength{\rcollength}{1.85in}%
%
\begin{tabular}[t]{@{}p{\textwidth-\rcollength}p{\rcollength}}
Moor House & \\
Roeburndale West & \\
Lancaster & +447907808969 \\
LA2 9LJ      & \href{mailto:will@kenyonmail.com}{will@kenyonmail.com}\\
\end{tabular}



\section{Higher Education}
\href{http://www.cam.ac.uk/}{\textbf{University of Cambridge}},  UK\hfill\textbf{2009--2013}

\subsection{M.Eng. \newline Computer Science \newline Class: Merit}
\textbf{Masters Project}: I implemented \textit{stack backtraces}, a frequently requested debugging feature for the Glasgow Haskell Compiler (GHC). Once my code makes it into a GHC release, it will make debugging easier for thousands of Haskell programmers.

\color{gray}
\vspace{0.5cm}
\textbf{Masters Courses}: Algebraic network routing, Automated reasoning, Category theory, Nominal sets, Complexity of logic.
\color{black}

\subsection{B.A. \newline Computer Science \newline Class: 2.1}
\textbf{Undergraduate Project:} Monte Carlo Tree Search (MCTS) is a state-of-the-art artificial intelligence technique, which has allowed computers to beat professional human players at \textit{Go}. I implemented the first MCTS library for \href{http://www.haskell.org/haskellwiki/Introduction}{Haskell}.

\vspace{0.5cm}
\textbf{Group Project:} \textit{FrontlineSMS:Radio} is an SMS hub which organizes messages received during a radio broadcast. Our group built the first prototype of FrontlineSMS:Radio. I personally delivered a demonstration of our tool to a packed out lecture theatre. The audience were invited to send text messages to my `radio show,' and I demonstrated our tool searching, and organizing their messages.

\vspace{0.5cm}
\color{gray}
\textbf{Undergraduate Courses:} Compilers, Decompilers, Graphics, Natural Language Processing, Digital Signal Processing, and Verification. Quantum Computing, and Security. 

\hspace{0.5cm}\textbf{Systems}: Computer Architectures, Networks, Operating Systems, and Electronics.  

\hspace{0.5cm}\textbf{Theory}:  Complexity, Databases, Semantics, Languages \& Automauta, and Types. 

\hspace{0.5cm}\textbf{Mathematics:} Logic, Discrete Mathematics, Probability, Calculus, Matricies, 

\hspace{0.5cm}Vectors, Fourier Series, Fourier Transforms, and Taylor Series. 

\hspace{0.5cm}\textbf{Physics:}  Electromagnetism, Relativity, and Quantum \& Classical Mechanics. 

\hspace{0.5cm}\textbf{Non-technical:} Buisness, E-commerce, Economics, Ethics, Law, and Philosophy.

\color{black}
\section{Secondary Education}\textbf{\href{http://www.queenelizabeth.cumbria.sch.uk/}{Queen Elizabeth School}}, Kirkby Lonsdale, Cumbria, UK \hfill \textbf{2002--2009}
\begin{innerlist}
%\item[] \textbf{GCE} Computing: \emph{A (598/600)}, Physics: \emph{A}, Maths: \emph{A}, Further Maths: \emph{A}. \hfill \textbf{2009}
%\item[] \textbf{GCE} Electronics: \emph{A (full marks in 5/6 modules)}\hfill \textbf{2008}
\item[] \textbf{GCE} Grade \emph{A} in Computing, Electronics, Physics, Maths, and Further Maths. \hfill \textbf{2009}
\item[] \textbf{GCSE} $9$\emph{A*}s and $4$\emph{A}s.\hfill \textbf{2007}

\end{innerlist}


\section{Employment}
%
\href{http://www.realvnc.com/}{\textbf{RealVNC Ltd.}},
Cambridge, UK \hfill \textbf{Summer 2011}

\begin{outerlist}
\item[] I started the \textbf{VNC Viewer for Google Chrome} project by building the first prototype. Since then, over 100,000 people have downloaded the production version.
\end{outerlist}
\blankline

\href{http://www.lancs.ac.uk/}{\textbf{Lancaster University}},
 UK        \hfill \textbf{Summer 2009 \& 2010}
\begin{outerlist}

\item[] The \textbf{\href{http://www.p2p-next.org/}{p2pnext}} project implemented p2p internet TV for the BBC and other broadcasters. I experimented with grouping geographically close peers to reduce load.
\item[] The \textbf{Community Wireless} project brought fast internet into a community with no broadband availability. I helped develop a \href{http://www.comp.lancs.ac.uk/resilience/project/wray-community-mesh-network}{wireless mesh network} based on \href{http://wiki.openwrt.org/about/start}{OpenWRT}. I worked on a range of tasks, from developing an access control portal, to developing a last-resort emergency recovery system, to fastening boxes to rooftops.
%\item Continued projects I had started in the previous summer using the documentation I had written.
%\item Installed the mesh boxes we had been working on all summer in a village and trained community volunteers to use the system and help others. Monitored usage using my monitoring software.


\end{outerlist}



%\textbf{\href{http://www.edfenergy.com}{EDF Energy}, Heysham II Nuclear Power Station}, Lancashire, UK\hfill \textbf{Summer 2008}
%\begin{outerlist}

%\item[] \textit{Work Experience}% **??? you arent saying anything. what was your title. this is already headed uder related employment. 
        
%\begin{innerlist}
%\item Gained understanding of how computer systems are used to monitor and control a nuclear reactor.
%\item Wrote code on development system (carbon copy of the computer system controlling the reactor). **this does not say anything to me. so what if it is a carbon copy. what good did you do? 

%\end{innerlist}

%\end{outerlist}

%

\subsection{\normalsize Skills \footnotesize Programming Languages}
\textbf{General}: \href{http://en.wikipedia.org/wiki/C_(programming_language)}{C}/\href{http://www.cplusplus.com/info/description/}{C$++$}, \href{http://www.haskell.org/haskellwiki/Introduction}{Haskell}, \href{http://java.sun.com/docs/overviews/java/java-overview-1.html}{Java}, \href{http://www.smlnj.org/sml.html}{ML}, \href{http://www.cse.unsw.edu.au/~billw/cs9414/notes/prolog/intro.html}{Prolog}, and \href{http://en.wikipedia.org/wiki/Visual_Basic}{Visual Basic} .

\textbf{Hardware Description}: \href{http://www.systemverilog.org/}{System Verilog}.

\textbf{UNIX Tools}: \href{http://www.gnu.org/software/bash/manual/html_node/What-is-Bash_003f.html}{Bash}, \href{http://cimss.ssec.wisc.edu/wxwise/class/aos340/spr00/whatismatlab.htm}{Matlab}, \href{http://perldoc.perl.org/perlfaq1.html}{Perl}.

\textbf{Web:} \href{https://developer.mozilla.org/en/JavaScript/Guide/JavaScript_Overview}{Javascript}, \href{http://en.wikipedia.org/wiki/HTML}{HTML}, \href{http://en.wikipedia.org/wiki/PHP}{PHP}, \href{http://perldoc.perl.org/perlfaq1.html}{Perl}. and \href{http://en.wikipedia.org/wiki/SQL}{SQL}.

I studied programming languages formally in \href{http://www.cl.cam.ac.uk/teaching/1112/Types/}{Type Theory} and  Semantics courses.

%\subsection{Code Style \& Documentation}
%I am versatile, and have can quickly adopt the coding/documentation style used by a particular company. I document code as I write it, and am used to using documentation generation tools such as \href{junit.com}{JUnit}.

\subsection{Libraries \& Tools}
\textbf{Concurrency:} Java concurrency, and \href{http://www.kernel.org/doc/man-pages/online/pages/man7/pthreads.7.html}{pthreads} in C/C$++$.

\textbf{Networking:} Java sockets, and \href{http://unixhelp.ed.ac.uk/CGI/man-cgi?select+2}{sockets} in C/C$++$.

\textbf{General:} Standard C/C$++$ library, \href{http://www.sgi.com/tech/stl/stl_introduction.html}{Standard Template Library}.

\textbf{Version Control:} \href{http://eagain.net/articles/git-for-computer-scientists/}{Git}, and \href{http://svnbook.red-bean.com/en/1.6/svn.intro.whatis.html}{Subversion}.

\textbf{Automated Documentation Generation:} \href{http://www.haskell.org/haskellwiki/Haddock}{Haddock}, and \href{http://www.devx.com/tips/Tip/13579}{JavaDoc}.

\textbf{Automated Unit Testing:}
\href{http://www.cse.chalmers.se/~rjmh/QuickCheck/manual_body.html}{QuickCheck}, and
\href{http://junit.sourceforge.net/doc/faq/faq.htm#overview_1}{JUnit}.



\textbf{Web:} \href{http://en.wikipedia.org/wiki/Ajax_(programming)}{Ajax}, \href{http://drupal.org/about/}{Drupal}, \href{http://nodejs.org/}{node.js}, and \href{http://www.websocket.org/}{Websocket}. 

\textbf{Operating Systems:} \href{http://en.wikipedia.org/wiki/Linux}{Linux}, \href{http://en.wikipedia.org/wiki/Macintosh}{Macintosh}, and \href{http://en.wikipedia.org/wiki/Windows}{Windows}. 

\subsection{Data Structures \& Algorithms}
\href{http://www-users.cs.umn.edu/~janardan/FHEAP/}{Fibonacci Heaps}, \href{http://en.wikipedia.org/wiki/Disjoint-set_data_structure}{Disjoint Sets}, \href{http://www.bluerwhite.org/btree/}{B Trees}, \href{http://en.wikipedia.org/wiki/Red–black_tree}{Red Black Trees}, \href{http://en.wikipedia.org/wiki/External_sorting}{External $n$-way Merge Sort}, and basic data structures \& algorithms. I have implemented many of the above, and I can find asymptotic complexity bounds for algorithms using standard and amortized analysis.

\subsection{Logic \& Discrete Mathematics}
I can prove statements, formally/informally in first/second/higher order logics, temporal logics, hoare/sep\-aration logic, polymorphic lambda calculi and sequent calculi. This formal background gives me the mind-set required to write good code.

%\subsection{Specification \& Verification}
%I can write specifications for program code, and prove** the code meets the specifications. I can model small software/hardware systems, and proove that the model meets a specification written in Computation Tree Logic, Linear Time Logic or PSL.

\section{\footnotesize Teamwork}
Teamworking skills were essential at internships, and during my undergraduate group project. I had client meetings, deadlines to meet, coding and documentation styles to adhere to. I like putting myself `out there,' if there is a presentation to be made, I'm always the first to volunteer.

%\subsection{Law}
%I have good knowledge of the laws and regulations relevant to the IT industry***. This was obtained during my undergraduate degree and internships***.

\section{Intrests}
%
During university I was seriously involved in rowing. I rowed in the Lightweight Boat Race against Oxford in 2013, and was a spare in 2012. In 2013, I also made it to the second round of trials for the Great Britain under-23 team.
I have also participated fully in society life, having skied (competitively), sailed (competitively), and sung (socially) my way through my free time at university.

%Ski at a local dry ski slope and participate in \href{www.ersa.com}{Eastern Region} races.**

%Sail a \emph{Laser I} dinghy competitively in weekly club races at a local sailing club.



%Have sung in \href{http://www.emmamusic.co.uk/index.php?page=chorus}{Emmanuel College Chorus}, performing in frequent concerts  and charity carols on street.

%Performed in \href{http://cambridge.gands.org.uk/}{Cambridge University Gilbert and Sullivan Society} operettas.**

%Danced with the \href{http://www.cambridgedancers.org/}{Cambridge Dancers Club}.**


\section{Referees}
%
\href{http://realvnc.com}{Hannah Clear} (\href{mailto:hannah.clear@realvnc.com}{hannah.clear@realvnc.com})

\href{http://www.cl.cam.ac.uk/~nad10}{Neil Dodgson} (\href{mailto:nad10@cam.ac.uk}{nad10@cam.ac.uk})

\href{http://www.comp.lancs.ac.uk/department/staff.php?name=race}{Nicholas Race} (\href{mailto:n.race@lancaster.ac.uk}{n.race@lancaster.ac.uk})

\end{document}

%%%%%%%%%%%%%%%%%%%%%%%%%% End CV Document %%%%%%%%%%%%%%%%%%%%%%%%%%%%%
